\section{Introduction}
\label{sec:intro}

We present a software implementation of the algorithm EASAL (Efficient Atlasing
and Search of Assembly Landscapes) \cite{Ozkan2011}. This implementation
generates, describes, and explores the feasible relative positions of two
point-sets $A$ and $B$ of size $n$ in $\mathbb{R}^3$ that are mutually
constrained by distance intervals. Formally, a Euclidean orientation-preserving
isometry $T \in SE(3)$ is \emph{feasible }%\jorg{for forming an assembly}} 
if, for $\emph{dist}_{a,b}$ defined as the Euclidean norm
$||a-T(b)||$, the following hold:

\begin{align}
%\begin{aligned}
\forall (a\in A, b\in B),\qquad& \emph{dist}_{a,b} \qquad \ge \rho_{a,b}& \tag{\distall}\label{eqn:constraints}\\
\exists (a\in A, b\in B),\qquad& \emph{dist}_{a,b} - \rho_{a,b} \le \delta_{a,b}, & \rho_{a,b}, \delta_{a,b} \in \mathbb{R}_+.\tag{\distexist}\label{eqn:preferredConstraints}
%\end{aligned}
\end{align}

\noindent Constraint \cone\ means that $T$ is infeasible when there exists a
pair $(a, T(b))$ that is too close. Constraint \ctwo\ implies that at least one
pair $(a,T(b))$ is within a \emph{preferred} distance interval.  Consider for
example, sets $A$ and $B$ of centers of non-intersecting spheres (see
\figref{fig:pctreeInput} and \figref{fig:3Input}). With $\rho_a$,
$\rho_b$ the sphere radii, the constant $\rho_{a,b}$ in \cone\ equals $\rho_a +
\rho_b$.  Note that the ambient dimension of Problem (\cone, \ctwo) is 6,
namely, the dimension of SE(3).  When $T$ is feasible, the \emph{Cartesian
configuration} $T(B)$ is called a \emph{realization} of the constraint system
(\cone, \ctwo).  When  $\delta_{a,b}\approx 0$ the effective dimension of the
realization space is 5.

The input to EASAL consists of up to four components. 
\begin{enumerate}
\item [--] $k=2$ point-sets $A$ and $B$ with $n$ points each. (The submitted
implementation is for two point-sets, but the theory and the algorithms
generalize to $k$ point-sets and ambient dimension $6(k-1)$)

\item[--] The pairwise distance interval parameters $\rho_{a,b}, \delta_{a,b} \in
\mathbb{R}_+$.

\item[--] Optional: global constraints imposed on the overall configuration. 

\item[--] Optional: a set of active constraints of interest. (Only constraint
regions including at least one of these active constraints is sampled and added
to the atlas.)
\end{enumerate}


The main output of EASAL is the dimensional, topological and geometric
structure of the \emph{realization space}, i.e., all $T(B)$ satisfying (\cone,
\ctwo). The realization space is represented as the \emph{sweep} of the
individual realizations (see \figref{fig:pctreeSweep} and
\figref{fig:RealizationSweepView}). The sweep representation shows $A$ together with all
feasible realizations $T(B)$ traced out.


To describe this space, EASAL employs three strategies.  First, EASAL
partitions the realization space into \emph{active constraint regions}, each
defined by the set of \emph{active constraints}, i.e., the pairs $(a,b)$
satisfying \ctwo.  These pairs are edges of the \emph{active constraint graph}
used to label the region.  Such a graph can be analyzed by generic
combinatorial rigidity theory \cite{CombinatorialRigidity}, in particular, the
co-dimension of an active constraint region (see Section
\ref{sec:stratification}) is typically the number of active constraint edges.
Since the active constraint regions satisfy polynomial equations and
inequalities, the realization space is semi-algebraic set (a union of sets
defined by polynomial inequalities).  This is the setting of a Thom-Whitney
stratification of semi-algebraic sets \cite{Kuo}.

Second, EASAL organizes and represents the active constraint regions in a
partial order (directed acyclic graph) so that the active constraint graph of a
region is a subgraph of the active constraint graph of its boundary regions.
This organization is called the \emph{atlas}. To construct the atlas, EASAL
recursively starts from the interior of an active constraint region and locates
boundary regions of strictly one dimension less. Such boundary regions
generically have exactly one additional active constraint and the active
constraint graph has one additional edge. Considering only boundary regions of
exactly one dimension less improves robustness over searching directly for
lowest-dimensional regions. We note that, when a new \emph{child} region of one
dimension less is found, all its higher dimensional \emph{ancestor} regions are
immediately discovered since they correspond to a subset of the active
constraints.  Therefore, even if a region is missed at some stage, it will be
discovered once any of its descendants are found, for example, through one of its siblings.

 
Third, to locate the boundary region satisfying an additional active
constraint, EASAL applies the theoretical framework developed in
\cite{SiGa:2010}. EASAL efficiently maps (many to one) a $d$-dimensional active
constraint region $R$ with active constraint graph $G$, to a convex region of
$\mathbb{R}^d$ called the \emph{Cayley configuration space} of $R$. Define 
a \emph{non-edge} of $G$ as a pair $(a,b)$ not connected by an edge in $G$.
The Cayley configuration space of $R$ is defined intuitively as the set of
\emph{realizable} lengths of $d$ chosen non-edges of $G$. The variables representing these 
non-edge lengths are called the \emph{Cayley parameters}. In what follows, we simply refer to the
non-edges as Cayley parameters.  Since the Cayley
configuration space is convex, it allows for efficient sampling and search. In
addition, it is efficient to compute the inverse map from each point in the
Cayley configuration space (a \emph{Cayley configuration}) to its finitely many
corresponding Cartesian realizations.  The Cayley configuration space of a
$d$-dimensional active constraint region $R$ is discretized and represented as a
$d$-dimensional grid.  The Cayley points adjacent to the lower dimensional boundary regions of $R$ are
highlighted in different colors (See \figref{fig:pctreeSpace}).


%This is in contrast to many other methods such as the Metropolis Monte Carlo
%sampling (MC) and Molecular Dynamics (MD), which can only make
%small/differential motions away from a current configuration. 

Efficiency, accuracy, and tradeoff guarantees have been formally established for
EASAL (see Section \ref{sec:complexity}). The total number of active constraint
regions in the atlas could be as large as $O(k^2 \cdot n^{12k})$.  The maximum
dimension of a region is $6(k-1)$. If $r$ regions of dimension $d$ have to be
sampled, EASAL requires time linear in $r$ and exponential in $d$.  EASAL can
explore assemblies up to a million regions for \emph{small assemblies} in a few
hours on a standard laptop (see Section \ref{sec:atlasandpath}. By small
assemblies we mean constraint problems with $n \le 5000$ and $k=2$; or $n\le 3$
and $k\le 18$.  Efficiency can improve significantly when the point-sets are
identical, by exploiting symmetries in the configuration space
\cite{sym8010005}. 



Section \ref{sec:results} surveys numerical experimental results from
\cite{Ozkan2014MainEasal}, for (i) generating the atlas, (ii) using the atlas
to find paths between active constraint regions and (iii) using the atlas to
find the neighbor regions of an active constraint region. We also survey
experimental results from \cite{Ozkan2014MC}, comparing the performance of
EASAL with Metropolis Markov chain Monte Carlo (MC) and from \cite{Wu2014Virus}
for EASAL predicting the sensitivity of icosahedral T=1 viruses towards
assembly disruption.\\


\noindent\textbf{Organization:} 
After briefly reviewing applications of EASAL to molecular and materials
modeling and related work, the remainder of the paper is organized as follows.
Section \ref{sec:theory} discusses the theory underlying EASAL. Section
\ref{algorithms} discusses the algorithmic ideas and implementation. Section
\ref{sec:results} surveys experimental results, Section \ref{architecture}
sketches the software architecture.

\subsection{Application to Molecular and Materials Modeling} 

EASAL provides a new approach to the longstanding challenges in molecular and
soft-matter self-assembly under short range potential interactions.  EASAL can
be used to estimate free-energy, binding affinity and kinetics. For example, EASAL
can be applied to (a) supramolecular self-assembly or docking starting from rigid
molecular motifs e.g., helices, peptides, ligands etc. or (b) self-assembly of
clusters of multiple particles each consisting of 1-3 spheres - e.g., in
amphiphiles, colloids or liquid crystals. 

In the context of molecular assembly, rigid components of the molecules
correspond to the input point-sets $A$ and $B$, and atoms correspond to the
points $a\in A$ and $b\in B$. The active constraint regions correspond to
regions of constant potential energy derived from discretized
Lennard-Jones~\cite{Jones463} potential energy terms. It is intractable or at
least prodigiously expensive to atlas large molecular assemblies by any naive
global method. Assemblies are typically recursively decomposed into smaller
assemblies (defined above) and recombined. Generally, the input molecules have
a small set of interfaces (pairs of atoms, one from each molecule) where bond
formation is feasible. These are given as input by specifying a set of active
constraints of interest corresponding to the interfaces. EASAL atlases only
those $r$ active constraint regions where at least one of these constraints is
active (i.e., $C_2$ holds).


\subsubsection{Geometrization of Molecular Interactions in EASAL} 
\label{sec:geometrization}
In EASAL, the inter-atomic Lennard-Jones potential energy terms are
\emph{geometrized} into 3 main regions: (i) large distances at which no force
is exerted between the atoms, such atom pairs, called inactive constraints,
correspond to $(a,b)$ such that $dist_{a,b} > \rho_{a,b} + \delta_{a,b}$, (ii)
very close distances that are prohibited by inter-atomic repulsion or
inter-atomic collisions and violating \cone. (iii) the interval between these,
known as the Lennard-Jones \emph{well}, in which bonds are formed,
corresponding to the preferred distance or active constraints defined in \ctwo.

The pairwise Lennard-Jones terms are typically input only for selected pairs of
atoms, one from each rigid component. Hard-sphere \emph{steric} constraints,
apply to all other pairs and enforce (i) and (ii) with $\delta_{a,b} = 0$ in
\ctwo.  Having more active constraints corresponds to lower potential energy,
as well as to lower effective dimension of the region. The lowest potential
energy is attained at zero-dimensional regions, i.e., for rigid active constraint
graphs and finitely many configurations. For each rigid active constraint graph
$G$, the corresponding potential energy \emph{basin} includes well-defined
portions of higher dimensional regions whose active constraint graphs are
non-trivial subgraphs of $G$. In this manner the Cartesian configuration space
is partitioned into potential energy basins. Free energy of a configuration
depends on the depth and weighted relative volume (configurational entropy) of
its potential energy basin.

Since lowest free energy corresponds to lowest potential energy \emph{and} high
relative volume of the potential energy basin, we are often specifically
interested in zero-dimensional regions where the potential energy is lowest.
However, the volume of the potential energy basins corresponding to these
regions typically include portions of all of their higher dimensional ancestor
regions. These ancestor regions should therefore found and explored. Similarly,
computing kinetics involves a comprehensive mapping of the topology of paths
between regions, where the paths could pass through other regions of various
effective dimensions. Although paths would be expected to favor low dimensional
regions since they have the lowest energy, these paths could be long, requiring
many energy ups and downs, as well as backtracking, which could cause more
direct paths to be favored that pass through higher dimensional, higher energy
regions. 


EASAL (i) directly atlases and navigates the complex topology of small assembly
configuration spaces (defined earlier), crucial for understanding free-energy
landscapes and assembly kinetics; (ii) avoids multiple sampling of
configurational (boundary) regions, and minimizes rejected samples, both
crucial for efficient and accurate computation of configurational volume and
entropy and (iii) comes with rigorously provable efficiency, accuracy and
tradeoff guarantees (see Section \ref{sec:complexity}). To the best of our
knowledge, no other current software provides such functionality.


\subsection{Related Work}
\subsubsection{Related Work on Geometric Algorithms} 
A generalization of Problem (\cone, \ctwo) arises in the robotics motion planning
literature with exponential time algorithms to compute a roadmap (a version of
atlas) and paths in general semi-algebraic sets
\cite{bib:canny-roadmap,canny-alg,basu}, with probabilistic versions to improve
efficiency \cite{kavraki1,kavraki2}. For the Cartesian configuration space of
non intersecting spheres, Baryshnikov et al. and Kahle characterize the
complete homology \cite{Baryshnikov08022013,Kahle2011}, viable only for
relatively small point-sets or spheres, while more empirical computational
approaches for larger sets \cite{PhysRevE,Bubenik10statisticaltopology} come
without formal algorithmic guarantees. A geometric rigidity approach was
primarily used to characterize the graph of contacts of arbitrarily large
jammed sphere configurations in a bounded region
\cite{Kahle2012,Connelly:Jamming}.

Unlike these approaches, the goal of EASAL is to describe the configuration
space of Problem (\cone, \ctwo). In addition, EASAL is
deterministic and its efficiency follows from exploiting special properties of
those semi-algebraic sets that arise as configuration spaces of point-sets
constrained by distance intervals. 

\subsubsection{Related Work on Molecular and Materials Modeling}
\label{ssub:MolecularRelatedWork} 
The simplest form of supramolecular self assembly and hence the simplest
application of Problem (\cone, \ctwo) is site-specific docking. Computational
geometry, vision and image analysis have been used in site-specific docking
algorithms \cite{Bespamyatnikh,Choi2004,pmid1549581,Duhovny2002,pmid15980490}.
Unlike the more general goals of EASAL, the goal of these algorithms is to
simply find site-specific docking configurations with optimal binding affinity.
While this depends on equilibrium free energy, docking methods simply
evaluate an approximate free energy function.

On the other hand, prevailing methods for direct free energy computation - that
must incorporate both the depth and relative weighted volumes (entropy) of the
free energy basin - use highly general approaches such as Monte Carlo (MC) and
Molecular Dynamics (MD) simulation. They deal with a notoriously difficult
generalization of Problem (\cone, \ctwo) 
~\cite{kaku,Andricioaei_Karplus_2001,Hnizdo_Darian_Fedorowicz_Demchuk_Li_Singh_2007,Hnizdo_Tan_Killian_Gilson_2008,Hensen_Lange_Grubmuller_2010,Killian_Yundenfreund_Kravitz_Gilson_2007,Head_Given_Gilson_1997,GregoryS201199,doi:10.1021/jp2068123}.
Ergodicity of these methods is unproven for configuration spaces of high
geometric or topological complexity with low energy, low volume regions (low
effective dimension) separated by high energy barriers. Hence they require
unpredictably long trajectories starting from many different initial
configurations to locate such regions and compute their volumes accurately.


While these methods are applicable to a wide variety of molecular modeling
problems, they do not 
take advantage of the simpler inter-molecular
constraint structure of assembly (\cone, \ctwo) compared to,
say, the intra-molecular folding problem (see \cite{assembly:folding}):
active constraint graphs that arise in assembly (see \figref{fig:3-trees}) 
yield convexifiable configuration spaces whereas
the folding problem has additional `backbone' constraints that prevent
convexification. 
Therefore, even though the energy functions used by
MC and MD can differ in assembly and folding, 
these methods miss out on critical advantages by not explicitly
exploiting special geometric properties of small assembly configuration spaces. 
EASAL on the other hand exploits such geometric properties 
via Cayley convexification. 

We do not review the extensive literature on (ab-initio) simulation or other
decomposition-based methods that are required to tractably deal with large
assemblies.  For small cluster assemblies from spheres, i.e., $n=1$ and $k\le
18$, there exist a number of methods to compute free energy and configurational
entropy of subregions of the configuration space
\cite{Holmes-Cerfon2013,Arkus2009,Wales2010,Beltran-Villegas2011,Calvo2012,Khan2012,Hoy2012,Hoy2014}.
Working with traditional Cartesian configurations, they must deal with
subregions that are comparable in complexity to the entire Cartesian
configuration space of small molecules such as cyclo-octane
\cite{Martin2010,Jaillet2011,Porta2007}. With $n=3$, there are bounds for
approximate configurational entropy using robotics-based methods without
relying on MC or MD sampling~\cite{GregoryS201199}. For arbitrary $n$ and
starting from MC and MD samples, recent heuristic methods infer a topological
roadmap
\cite{Gfeller_DeLachapelle_DeLos_Rios_Caldarelli_Rao_2007,Varadhan_Kim_Krishnan_Manocha_2006,Lai_Su_Chen_Wang_2009,10.1371/journal.pcbi.1000415}
and use topology to guide dimensionality reduction
\cite{Yao_Sun_Huang_Bowman_Singh_Lesnick_Guibas_Pande_Carlsson_2009}. In
particular \cite{Holmes-Cerfon2013} formally showed that their (and EASAL's)
geometrization is physically realistic, but, they directly search for
hard-to-find zero dimensional active constraint regions by walking
one-dimensional boundary regions of the Cartesian configuration space. In
addition they compute one and two dimensional volume integrals. 

To the best of our knowledge these methods do not exploit key features of
assembly configuration spaces that are crucial for EASAL's efficiency and
provable guarantees. These include Thom-Whitney stratification, generic
rigidity properties, Cayley convexification, and recursively starting from the
higher-dimensional interior and locating easy-to-find boundary regions of
exactly one dimension less. Using these and adaptive Jacobian sampling
\cite{Ozkan2014Jacobian}, EASAL can rapidly find all generically
zero-dimensional regions and can be used to compute not only one and two, but
also higher dimensional volume integrals \cite{Ozkan2014MainEasal}, as well as
paths that pass through multiple regions of various dimensions. This is
important for free energy and kinetics computation.


\subsubsection{Recent Work Leveraging EASAL}
\label{related}
EASAL variants and traditional MC sampling of the assembly landscape of two
transmembrane helices have recently been compared from multiple perspectives in
order to leverage complementary strengths \cite{Ozkan2014MC}. In addition,
EASAL has been used to detect assembly-crucial inter-atomic interactions for
viral capsid self-assembly~\cite{Wu2012,Wu2014Virus} (applied to 3 viral
systems: Minute Virus of Mice (MVM), Adeno-Associated Virus (AAV), and
Bromo-Mosaic Virus (BMV)). This work exploited symmetries and utilized the
recursive decomposition of the large viral capsid assembly into an assembly
pathway of smaller assembly intermediates. Adapting EASAL to exploit symmetries
was the subject of \cite{sym8010005}.


Though the submitted implementation can handle only two point-sets as input
($k=2$), for greater than 2 point-sets, the extension of the EASAL algorithm
and implementation have been shown to be straightforward
\cite{OzSi:2010,Ozkan2014MainEasal}. When $n=1$, i.e., each point-set is an
identical singleton sphere, exploiting symmetries leads to simpler computation.
EASAL has been used to compute 2 and 3 dimensional configurational volume
integrals for 8 assembling spheres for the first time
\cite{Ozkan2014MainEasal}, relying on Cayley convexification. Building upon the
current software implementation of EASAL, an adaptive sampling algorithm
directly leads to accurate and efficient computations of configurational region
volume and path integrals \cite{Ozkan2014Jacobian}.
