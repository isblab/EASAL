\section{Conclusion}
\label{conclusion}
The EASAL software generates, describes, and explores key aspects of the
topology and geometry of the configuration space of point-sets in
$\mathbb{R}^3$. To achieve this, it uses three strategies, (i) EASAL partitions
the realization space into active constraint regions each defined by the set of
active constraints. The graph of active constraints called the active
constraint graph is then used for analysis using generic combinatorial rigidity
theory.  (ii) EASAL organizes the active constraint regions in a partial order
called an atlas which establishes a parent child relationship between active
constraint regions that generically differ by exactly one active constraint.
To build the atlas, EASAL starts from the interior of an active constraint
region and recursively finds boundaries of one dimension less.  (iii) To locate
the boundary region satisfying exactly one additional constraint, EASAL uses
the theory of Cayley convexifiability to map (many to one) a $d$-dimensional
active constraint region to a convex region in $\mathbb{R}^d$ called the Cayley
configuration space of the region.  This allows for efficient sampling and
search. In addition, it is efficient to compute the inverse map from each point
in the Cayley configuration space to its finitely many Cartesian realizations.
With EASAL we obtain formal guarantees for quantitative accuracy and running
times.  The EASAL software optionally provides a GUI which can be used for
intuitive visual verification of results.

In the context of molecular assembly, EASAL distinguishes assembly from other
processes such as folding in that assembly admits to Cayley convexification of
active constraint regions. More general methods like MC and MD, though
applicable to a wider variety of molecular modeling problems, do not make this
distinction and hence are not as efficient as EASAL in the context of molecular
assembly.  For the problem of assembly, EASAL (i) directly atlases and
navigates the complex topology of small assembly configuration spaces, crucial
for understanding free-energy landscapes and assembly kinetics; (ii) avoids
multiple sampling of configurational (boundary) regions, and minimizes rejected
samples, both crucial for efficient and accurate computation of configurational
volume and entropy and (iii) comes with rigorously provable efficiency,
accuracy and tradeoff guarantees. To the
best of our knowledge, no other current software provides such functionality.


The paper reviews the key theoretical underpinnings, major algorithms and their
implementation; outlines the main applications such as computation of
free energy and kinetics of assembly of supramolecular structures or of
clusters in colloidal and soft materials; and surveys select
experimental results and comparisons.  
