\section{Algorithmic Implementation of Concepts}
\subsection{Exploration}
This method implements the exploration of the configuration space
discussed previously. When the current active constraint graph $G_H$
is minimally rigid, the stratum is $d=0$, i.e., there is no free parameter
left. Hence it will stop discovering new active constraint regions.
Otherwise, it exhaustively explores the region $R_{G_H}$ parametrized by $F$.
It samples uniformly with a given step size within the exact convex cover of the
Cayley parametrized regions. Exploration (binary search) is required in order
to detect the boundary of lower-dimensional regions. If a new constraint
becomes active, then a child \textit{atlas} node is created labeled
by $G_H \cup \{e\}$ and sampleExplore is recursively called to explore a new
AtlasNode and discover the corresponding lower dimensional region.\\


\begin{algorithm} [htbp]
 \SetKwInOut{Input}{input}\SetKwInOut{Output}{output}

 {\bf sampleExplore}\\
 \Input{atlasNode}
 \Output{A new cayleyPoint}
 \BlankLine

	let $G_H$ be the activeConstraintGraph of the atlasNode with $H$ being active constraints\\
	\If{ $G_H$ is minimally rigid}
		{stop;	}
	find the parameters $F$ so that $G = (V, E = F\cup H)$ is minimally rigid, i.e., complete 3-tree\\
	compute the convexChart for the parameters $F$\\



	\For{ each cayleyPoint $p$ within convexChart }
	{
		compute orientations of $p$ for each flip\\

		\For{ each orientation $o$ }
		{

			compute cartesianRealization $r$ of the entire rigid molecule from $o$;\\
			violated = ConstraintCheck($r$)
			\If{!violated}
			{

				add $o$ to $p$;\\
				get witness to boundary region if available through boundaryDetection(p);\\
				\If{ a new constraint $e$ becomes active in witness point from collection C }
				{
					$G'$ := $G_H \cup \{e\}$ ;\\
					\If{ $G'$ is not already present in the current atlas}
					{
						create child atlasNode labeled by $G'$\\
						save the child atlasNode to atlas\\
						sampleExplore(child);\\
					}
				}
			}
		}

		save $p$ to activeConstraintRegion of the activeConstraintGraph\\
	}

	\caption{sampleExplore}
\label{alg:sampleExplore}
\end{algorithm}

\subsection{A Posteriori Boundary Detection}
This implements the aposteriori boundary detection discussed earlier.
It is called as a subroutine from the `sampleExplore' method. In
order to detect the (lower-dimensional) boundaries that always lie between a
feasible region and a violated region, for each feasible Cayley point we check
all Cayley parameter directions to see if the neighboring point is infeasible
or not. If it is, then binary search will take place to find out the
boundary point with the new active constraints. A boolean map of the grid that stores
the statuses (feasible/violated/infeasible) of all neighboring points, would
eliminate recreating neighbor points. However the size of the map would be
prohibitive for nodes of higher dimension.

\begin{algorithm} [htbp]
 \SetKwInOut{Input}{input}\SetKwInOut{Output}{output}

 {\bf boundaryDetection}\\
 \Input{cayleyPoint}
 \Output{cayleyPoint}
 \BlankLine

	\For{ each neighboring point $p_n$ around p }
	{
		\If{ $p_n$ is violated by sterics}
		{
			// find boundary region in between $p$ and $p_n$ using binary search \\

			step\_size = step\_size/2;\\
			witness = walk back through $p$ with step\_size ;\\

			\While{ no new constraints in $G$ have become active }
			{
				step\_size = step\_size/2; 	\\
				\eIf{ witness is violated by sterics }
				{	witness = walk through $p$ with step\_size }
				{	witness = walk through $p_n$ with step\_size }
			}
		}
	}
	return witness;\\

 \caption{ boundaryDetection
\label{alg:boundaryDetection}}
\end{algorithm}

%--------------
%(SpaceView -> ChartView)
%is it good to keep the solver and data structure in same class, as in ACG or cayleyparametrization?
%-------------

