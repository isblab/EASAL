\subsection{Algorithms Implemented in EASAL}
\label{sec:algorithms}



\textbf{sampleExplore}:
The algorithm captures, stores, and labels the regions of the stratification of the configuration space.
The regions of the atlas are stored as nodes of a directed acyclic graph,  whose edges represent immediate containment or reachability.
Each region of the atlas is identified by a (small) \textit{activeConstraintGraph} $G_F$ where $F$ is the set of active constraints.
Newly discovered $G_F$ is tested to ensure that it is not already present in the current \emph{atlas}.
Only if $G_F$ is new is the region further explored; by default, this is done depth first.
New active constraints and regions are added one by one.
It samples the space at a desired level of refinement.\\

\ruijin{(I would expect the convex parametrization(includes ordering of parameters) and chart range computation to be put here. )}

%--------------------
\textbf{detectActiveConstraint}: Detects newly active constraint, and a new region of the stratification in a stratum of smaller dimension. EASAL\ relies on grid-sampling to find the children of each active constraint region and hence to uncover the topology of the atlas.
It exhaustively explores boundaries with step size $h$. When a new constraint becomes active, it recursively samples and explores the regions of its children.
%
However, the boundary detection is not guaranteed by grid-sampling alone.
The threshold may not be large enough to allow ConstraintCheck to define a close by atom pair to be a distance constraint. Furthermore, neighboring Cayley points may be in violation of steric constraints.
As a result, the boundary region would be missed since it is located in between a feasible point and a point violated by sterics. Exploration (by the way of binary search) is required to find any additional constraints that restrict the region, e.g.\ steric constraints that render a configuration infeasible.
